\lhead{\emph{Overview}}

% ***************************************************************************
\chapter{Overview}
% ***************************************************************************
% Explicar un poco que contiene este capítulo.



% ***************************************************************************
\section{Introduction}
% ***************************************************************************
Nowadays it is easy to see that with the rise of the robotics industry many manufacturing companies are and will continue developing robots aimed to take on a wide range of specific tasks. While waiting for a future standard robust multitask robot able to perform in complex daily life situations, all these robots are bound to work in environments where the interaction and collaboration among them is a common and necessary practice in order to effectively relieve human beings from repetitive and time consuming tasks.
Moreover it is very likely that they will work not only in collaboration with other robots but also alongside humans which, as unpredictable as we can be, not always will desire to be helped or even slightly assisted at all. Hence the ideal solution would imply flexible collaboration between humans and groups of robots.

% ***************************************************************************
\section{Motivation}
% ***************************************************************************
When a person with reduced mobility faces a task as apparently simple as moving objects from one place to another, such task may become a tedious and unnecessarily time consuming one, if not unfeasible at all. The presence of an artificial agent can ease the process but given the unstructured nature of a physical environment it is most likely going to encounter unexpected situations where it will not be able to provide the proper assistance. Hence an agent or group of agents able to adapt and act accordingly in different configurations of the environment would make for an adequate assistive solution.

% ***************************************************************************
\section{Problem Statement}
% ***************************************************************************
To program an agent, or group of agents, to be able to provide an efficient assistance in every single situation that may arise does not constitute a smart approach. It would be impractical, for obvious workload reasons. Thus, the problem at stake is a different one.
It is required to design a scalable framework for creating artificial agents able to provide assistance to humans in routine tasks. It must be able to adapt to unexpected environment difficulties. That is, to generate a plan of action that reaches a given specific goal with efficiency (understood in terms of time consumption) and regardless of the availability of the involved agents or the setting of its unstructured environment. That is of course, provided the involved agents have the required capabilities.

% ***************************************************************************
\section{Hypothesis}
% ***************************************************************************
Given the aforementioned problem, the Bratman's Believe-Desire-Intention (BDI) theory constitutes an appropriate software model to design the required agent framework.\par
It allows for a constant perception of the environment so the agent can keep track of any changes in it. With this the agent can generate plans of action and correct them according to the observed changes. The knowledge of the environment settings and the capabilities of the agent itself, as well as the capabilities of any other present agents, constitutes the BELIEF module.\par 
For an agent to execute a series of actions that will end up with the accomplishment of an issued objective task, such task must be first subdivided into a series of subgoals, that may or may not require to be reached in a sequential order. Each subgoal will then be reached by means of an action plan. The commitment to see fulfilled a given objective task by keeping its subgoals fulfilled is managed by the DESIRE module.\par
When dealing with each one of the subgoals that make up the objective task the agent must decide how to proceed. Taking into account the information provided in the belief module it will decide if it must be reached or not (should it were already reached), and in case of proceeding with the reaching either do it with the collaboration of the human or not. The INTENTION module will deal with this process.\par
In order to set up a multi-agent system with this framework, the agents will differ slightly in their programming. Each one of them will have knowledge of the capabilities that the rest of the agents have plus the capabilities that the considered agent itself has in a separate category.

% ***************************************************************************
\section{Objectives}
% ***************************************************************************
Considering what has already been said we could sum up the global objective of the project as the following: develop an agent architecture in order to implement it in a system with several agents that will assist humans in their daily tasks. The agents will receive a given objective task and will cooperate to carry it out in the most efficient way possible, considering also a human user with a variable degree of involvement in the task.
This general objective will entail the following specific objectives:\par
\begin{itemize}
\item Create a base agent with the BDI paradigm from which develop future agents.
\item Differentiate the base agent into specific agents.
\item Perform successful tests in a virtual environment.
\item Adapt the virtual agents to real robots.
\item Perform successful tests in a real environment.
\end{itemize}

% ***************************************************************************
\section{Methodology}
% ***************************************************************************
In order to achieve this objectives, the work to be done will be structured in two main tasks subdivided in several milestones:

\textit{\textbf{1 - Virtual implementation:}} 
\begin{itemize}
\item \textbf{Coding of agent0:} coding of the BDI framework as a basic virtual agent and implemented as a ROS node that will include the essential functions that every agent must have to interact with its environment.
\item \textbf{Coding of specific agents:} once the agent0 is ready it will be copied as many times as agents are present in the real environment. Each copy will be modified to include the specific abilities that the corresponding robot will have.
\item \textbf{Testing phase 1:} testing of the multi-agent system considering virtual environment.
\end{itemize}
    
\textit{\textbf{2 - Real implementation:}} 
\begin{itemize}
\item \textbf{Adaptation to real robots:} the virtual system will be linked to the real environment. The inputs and outputs that the agents as nodes will deal with won't come now from (and go to) a virtual environment but from the real workspace where the robots interact.
\item \textbf{Testing phase 2:} testing of the fully implemented multi-agent system considering pick and place tasks with several graspable objects in a domestic environment. 
\end{itemize}

% ***************************************************************************
%\section{Document Structure}
% ***************************************************************************